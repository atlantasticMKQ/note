% Created 2019-09-09 一 08:52
% Intended LaTeX compiler: xelatex
\documentclass[11pt]{article}
\usepackage{graphicx}
\usepackage{grffile}
\usepackage{longtable}
\usepackage{wrapfig}
\usepackage{rotating}
\usepackage[normalem]{ulem}
\usepackage{amsmath}
\usepackage{textcomp}
\usepackage{amssymb}
\usepackage{capt-of}
\usepackage{hyperref}
\usepackage[scheme=plain]{ctex}
\author{MKQ}
\date{\today}
\title{Physical Chemistry 03}
\hypersetup{
 pdfauthor={MKQ},
 pdftitle={Physical Chemistry 03},
 pdfkeywords={note},
 pdfsubject={},
 pdfcreator={Emacs 24.5.1 (Org mode 9.0.3)}, 
 pdflang={English}}
\begin{document}

\maketitle
\tableofcontents

\section{气体分子动理论 <现在基本不太用了>}
\label{sec:orgbf00a09}
只能算平动,现在转动什么的都可以算
气体分子在*经典力学*下的*统计平均描述*
\subsection{气体压强的统计解释}
\label{sec:org7547d02}
\begin{itemize}
\item 选一个面元
\item 气体分子质量:m
\item 选择一群速度为v\(_{\text{i}}\) ,数量为n\(_{\text{i}}\) 的分子
\item 计算它们撞到器壁上的动量改变
\end{itemize}
\[
\Delta I= \sum 2mn_i v_{ix}^2 \Delta S =nm\bar{v_{x}^2}\Delta S
\]
\begin{itemize}
\item 然后计算力
\item 计算所有分子<其中一半是反方向的所以再除以二>
\end{itemize}
\[
\bar{v_x^2}=\bar{v_y^2}=\bar{v_z^2}=\frac{1}{3} v^2
\]
\[
E_k =\frac{1}{2} m v^2
\]
\[
p=\frac{2}{3} n \bar{E_k}
\]
\subsection{气体温度的统计解释}
\label{sec:orgdf3cbfa}
\[
pV=\frac{N}{N_A}RT=NkT
\]
\[
E_k=\frac{3}{2} kT
\]
温度T是运动分子平均动能的量度
\subsubsection{均方根速率(root mean square rate)}
\label{sec:org6e63dcc}
\[
v_rms =\sqrt{\frac{3kT}{m}}=\sqrt{\frac{3RT}{M}}
\]
和温度直接关联
\section{Maxwell分布和应用}
\label{sec:orga4c235c}
\subsection{统计规律性和概率分布}
\label{sec:orga83f9f5}
\begin{itemize}
\item 平衡态理想气体热运动速率在(v\textasciitilde{}v+dv)内的概率
\item 
\end{itemize}
\[
P(v~v+dv)=\frac{dN}{N}=f(v)dv
\]
\begin{itemize}
\item 概率密度函数
\end{itemize}
\[
f(v)=\frac{P(v~v+dv)}{dv}=\frac{dN}{Ndv}
\]
出现在v附近单位速率空间内的概率,就是理想气体的概率分布函数
\begin{itemize}
\item 满足归一化条件
\end{itemize}
定义域积分为1
$\backslash$[

$\backslash$]
\subsubsection{可以解决的问题}
\label{sec:orgd47439c}
\begin{itemize}
\item 速率在v附近dv的间隔内的分子数
\end{itemize}
\[
dN=Nf(v)
\]
\begin{itemize}
\item 一个与速率有关的函数F(v)的平均值
\end{itemize}
\[
\bar{F(v)}=\frac{1}{N}\sum F(v)dN =\sum F(v)f(v)dv
\]
\subsubsection{人物}
\label{sec:orgb140d7e}
\begin{itemize}
\item Maxwell
\item Boltzman <S=kln\(\Omega\)>
\end{itemize}
\subsection{Maxwell分布律}
\label{sec:orge7e9802}
任意速度分量独立而且相同
\[
f(矢量(v))=f(v_x ,v_y ,v_z )=f(v^2 )= f(v_x )f(v_y )f(v_z )
\]
然后两边对v\(_{\text{x}}\) 求导
\[
\frac{df(v_x )}{f(v_x )}=-\beta dv_x^2
\]
\[
f(v_x )=C_1 e^{-\beta v_x^2 }
\]
就得到
\[
f(矢量(v))=f(v_x ,v_y ,v_z )=Ce^{-\beta(v_x^2 + v_y^2 + v_z^2 )}
\]
由椭圆积分带入归一化条件解\(\beta\)
\[
C=(\frac{\beta}{\pi})^{\frac{3}{2}}
\]
还有

\[E_k=\frac{3}{2}kT\]
\[
E_k=\frac{1}{2} m \iiint
 (v_x^2 ,v_y^2 ,v_z^2)f(v_x ,v_y ,v_z)dv_x dv_y dv_z
\]
\subsection{Maxwell速率分布公式}
\label{sec:org8f41aff}
\[
f((v))=(\frac{m}{2\pi kT})^{\frac{3}{2}} e^{-\frac{1}{2}mv^2/kT}
\]
\subsubsection{坐标变换把它变成球坐标}
\label{sec:org3c71a00}
\[
d(v)=dv_x dv_y dv_z =v^2 \sin\rho dv d\rho d\phi
\]
\[
f((v))d(v)=f(v,\rho ,\phi )v^2 sin\rho dv d\rho d\phi =4\pi (\frac{m}{2\pi kT})^{\frac{3}{2}} e^{-\frac{1}{2}mv^2 /kT}v^2dv
\]
\end{document}
