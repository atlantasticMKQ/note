% Created 2019-09-16 一 08:31
% Intended LaTeX compiler: xelatex
\documentclass[11pt]{article}
\usepackage{graphicx}
\usepackage{grffile}
\usepackage{longtable}
\usepackage{wrapfig}
\usepackage{rotating}
\usepackage[normalem]{ulem}
\usepackage{amsmath}
\usepackage{textcomp}
\usepackage{amssymb}
\usepackage{capt-of}
\usepackage{hyperref}
\usepackage[scheme=plain]{ctex}
\author{MKQ}
\date{\today}
\title{Physical Chemistry 05}
\hypersetup{
 pdfauthor={MKQ},
 pdftitle={Physical Chemistry 05},
 pdfkeywords={note},
 pdfsubject={},
 pdfcreator={Emacs 24.5.1 (Org mode 9.0.3)}, 
 pdflang={English}}
\begin{document}

\maketitle
\tableofcontents

\section{热力学能}
\label{sec:orgbc1b279}
\subsection{系统的总能量}
\label{sec:org1bcde99}
\[
E=E_T +E_V +U
\]
\begin{itemize}
\item E:体系总能量
\item E\(_{\text{T}}\) :系统整体的平动能
\item E\(_{\text{V}}\) :系统在外场中的位能
\end{itemize}
\subsection{热力学能}
\label{sec:org06c7096}
是系统状态的函数
包括系统内分子的平动能,转动能,振动能(和分子内部的化学键相关)
\[
3N-3-2/3(减去三个平动自由度,线性分子两个转动自由度,其他三个,剩下都是震动的自由度)
\]
还有电子运动能量,核能,分子间相互作用势能
\subsection{热力学第一定律}
\label{sec:orgeb9e0cb}
\[
\Delta U=Q+W
\]
\[
dU=\delta Q+\delta W
\]
适用于封闭的系统
热力学能是状态函数,在定态下有定值
热力学能的绝对值无法确定,但是可以关注变化量
\[
U=U(T,V)
\]
有着全微分的性质
\begin{itemize}
\item 将无法测量的U转化为可以测量的Q和W
\end{itemize}
\subsubsection{注意}
\label{sec:orgf2c5f30}
\begin{itemize}
\item 隔离系统的热力学能是一个常数U
\item 等容过程W=0
\item 绝热过程W=\(\Delta\) U
\item W和Q不是状态函数,会随着状态变化的途径变化而变化
\item 功和热是改变内能唯二的方式
\end{itemize}
\subsubsection{功和热}
\label{sec:orgeb7b267}
\begin{itemize}
\item 功:大量质点以有序的方式传递的能量
\item 热:以无序运动的方式传递的能量
\end{itemize}
\subsection{功与可逆途径}
\label{sec:orgf6ac790}
\subsubsection{膨胀功}
\label{sec:orgbe9873c}
由于体积变化而做的功
\begin{itemize}
\item 这里举的是刚性圆筒活塞往活塞上扔沙子的操作
\item 这里做功的是外压
\end{itemize}
\[
\delta W_e =-p_eAdz=-p_edV
\]
\begin{enumerate}
\item 自由膨胀
\label{sec:orge1cac0a}
外压为零,不对外做功
\item 等外压膨胀
\label{sec:orgc7f3c52}
这个也不可逆
\item 多次等外压膨胀
\label{sec:org6e5beb6}
\item 可逆过程
\label{sec:org7a09ac8}
内压始终等于外压,而且二做的功最大
\begin{itemize}
\item 也叫作准静态过程
\end{itemize}
\end{enumerate}
\end{document}
