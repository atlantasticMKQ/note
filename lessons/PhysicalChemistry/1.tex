% Created 2019-09-04 三 09:31
% Intended LaTeX compiler: xelatex
\documentclass[11pt]{article}
\usepackage{graphicx}
\usepackage{grffile}
\usepackage{longtable}
\usepackage{wrapfig}
\usepackage{rotating}
\usepackage[normalem]{ulem}
\usepackage{amsmath}
\usepackage{textcomp}
\usepackage{amssymb}
\usepackage{capt-of}
\usepackage{hyperref}
\usepackage[scheme=plain]{ctex}
\author{MKQ}
\date{\today}
\title{Physical Chemistry 01}
\hypersetup{
 pdfauthor={MKQ},
 pdftitle={Physical Chemistry 01},
 pdfkeywords={phsical-chemistry note},
 pdfsubject={},
 pdfcreator={Emacs 24.5.1 (Org mode 9.0.3)}, 
 pdflang={English}}
\begin{document}

\maketitle
\tableofcontents

\section{预习}
\label{sec:org2fbe567}
\subsection{气体动理论基本公式的推导}
\label{sec:orgcee8dc1}
在一个分子总数为N,体积为V的密闭容器中,每个气体分子的质量为m,
将气体分子分成若干群(根据速度的不同),设各群分子的速度分别为,
u\(_{\text{1}}\) ,u\(_{\text{2}}\) ,u\(_{\text{3}}\) \ldots{}u\(_{\text{i}}\) \ldots{}单位体积内各群分子的数目分别为n\(_{\text{1}}\) ,n\(_{\text{2}}\) ,
n\(_{\text{3}}\) \ldots{}n\(_{\text{i}}\) \ldots{}这些数目的和即为单位体积内分子的数目n(n=N/V)

考虑第i群的分子,它的速度是u\(_{\text{i}}\) ,数密度为n\(_{\text{i}}\) ,将它的速度沿x,y,z,
轴进行分解,得到u\(_{\text{ix}}\),u\(_{\text{iy}}\),u\(_{\text{iz}}\),其中有
u\(_{\text{i}}^{\text{2}}\) = u\(_{\text{ix}}^{\text{2}}\) +u\(_{\text{iy}}^{\text{2}}\) + u\(_{\text{iz}}^{\text{2}}\) (这个结论之后会有用)。

再考虑一下单位时间内碰到器壁的分子数,假设第i群分子是朝向器壁的,
这样就可以碰上去了,那么单位时间dt碰到器壁dA上的分子数为
dA dt u\(_{\text{ix}}\) n\(_{\text{i}}\) (这里假设器壁与x轴垂直)
那么单位时间内的动量改变量为
\(\Delta\) p\(_{\text{i}}\) = 2m dA u\(_{\text{ix}}^{\text{2}}\) dt n\(_{\text{i}}\)
由于只有一半分子是朝向器壁的,所以把2拿掉,然后带入上面那个式子
P=1/3 m\(\sum\) u\(_{\text{i}}^{\text{2}}\) n\(_{\text{i}}\)
定义均方根速率,u\(^{\text{2}}\)=\Singma u\(_{\text{i}}^{\text{2}}\) n\(_{\text{i}}\)/n
于是
P=1/3 mu\(^{\text{2}}\) n
\subsection{压力和温度的统计概念}
\label{sec:orga095b08}
分子的平均平动能E\(_{\text{t}}\)=1/2 mu\(^{\text{2}}\) (u是均方根速率)
平均平动能与温度具有平行的关系
1/2mu\(^{\text{2}}\) =f(T)
分子的平均平动能是温度的函数
温度反映了大量分子无规则运动的剧烈程度,讨论几个分子的温度是没有意义的

\subsection{理想气体的状态方程}
\label{sec:org1a727c5}
V=f(p,T,N)
(然后对V作全微分)

pV=nRT
其中,玻尔兹曼常数k\(_{\text{a}}\)=R/N\(_{\text{A}}\)
\subsection{分子平均平动能与温度的关系}
\label{sec:orge964fd6}
pV=1/3 Nmu\(^{\text{2}}\)

pV=Nk\(_{\text{a}}\) T

E\(_{\text{t}}\)=1/2mu\(^{\text{2}}\)

E\(_{\text{t}}\)=2/3k\(_{\text{a}}\) T
\subsection{气体运动的速率分布}
\label{sec:orgbac7895}
这里需要很长的推导
\end{document}
