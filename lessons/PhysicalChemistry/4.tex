% Created 2019-09-11 三 10:14
% Intended LaTeX compiler: xelatex
\documentclass[11pt]{article}
\usepackage{graphicx}
\usepackage{grffile}
\usepackage{longtable}
\usepackage{wrapfig}
\usepackage{rotating}
\usepackage[normalem]{ulem}
\usepackage{amsmath}
\usepackage{textcomp}
\usepackage{amssymb}
\usepackage{capt-of}
\usepackage{hyperref}
\usepackage[scheme=plain]{ctex}
\author{MKQ}
\date{\today}
\title{Physical Chemistry 04}
\hypersetup{
 pdfauthor={MKQ},
 pdftitle={Physical Chemistry 04},
 pdfkeywords={note},
 pdfsubject={},
 pdfcreator={Emacs 24.5.1 (Org mode 9.0.3)}, 
 pdflang={English}}
\begin{document}

\maketitle
\tableofcontents

\section{Boltzmann 分布和 M-B 分布}
\label{sec:org18c2bc0}
考虑外场影响下的分子空间位置分布
达到平衡态的理想气体分子空间位置的分布情况
\subsection{气体分子在重力场下的分布}
\label{sec:org33468be}
在某一个高度z的附近取一个dz,然后这上下的分子有一个压力差
上面是p下面是p+dp
\[
n(z)mg\Delta S dz
\]
这一部分分子受到的重力如上
\[
dp=-n(z)mgdz
\]
受力平衡的条件,
\[
dp=kTdn
\]
理想气体状态方程
于是两边积分,就得到了
\[
n(z)=n_0 e^{-\frac{mgz}{RT}}
\]
\subsection{Boltzmann分布律}
\label{sec:orgfdac880}
\[
E_p =mgz
\]
然后推广到所有的保守立场
\[
n(z)=n_0 e^{-\frac{E_p}{RT}}
\]
\subsubsection{气体分子的分布函数}
\label{sec:org2e00f86}
\[
N=\iiint_V n(r) dV 
\]
\subsubsection{概率密度函数}
\label{sec:orgb797a56}
\[
f(x,y,z)dxdydz=\frac{dN}{N}=\frac{n_0}{N}e^{-\frac{E_p (x,y,z)}{kT}}
\]
\subsection{Maxwell-Boltzmann分布}
\label{sec:org49a1510}
\subsubsection{Maxwell}
\label{sec:orgacc2218}
\[
f(v)=\frac{xxx}{xxx} e^{-\frac{E_k}{kT}}
\]
\subsubsection{合成}
\label{sec:org091f917}
\begin{itemize}
\item 相空间,有六维,(x,y,z,v\(_{\text{x}}\) ,v\(_{\text{y}}\) ,v\(_{\text{z}}\) )
\item 代表点:单个分子某一时刻运动的点
\item 速度分布和位置分布相互独立
\end{itemize}
\subsubsection{注意}
\label{sec:orga21e0b5}
\begin{itemize}
\item 总能量是势能和平动动能
\item 对于多原子分子,还要考虑转动动能,振动,相互作用能
\end{itemize}
\end{document}
