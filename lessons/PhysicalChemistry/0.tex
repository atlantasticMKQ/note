% Created 2019-09-04 三 09:24
% Intended LaTeX compiler: xelatex
\documentclass[11pt]{article}
\usepackage{graphicx}
\usepackage{grffile}
\usepackage{longtable}
\usepackage{wrapfig}
\usepackage{rotating}
\usepackage[normalem]{ulem}
\usepackage{amsmath}
\usepackage{textcomp}
\usepackage{amssymb}
\usepackage{capt-of}
\usepackage{hyperref}
\usepackage[scheme=plain]{ctex}
\author{MKQ}
\date{\today}
\title{Physical Chemistry 00}
\hypersetup{
 pdfauthor={MKQ},
 pdftitle={Physical Chemistry 00},
 pdfkeywords={physical-chemistry note},
 pdfsubject={},
 pdfcreator={Emacs 24.5.1 (Org mode 9.0.3)}, 
 pdflang={English}}
\begin{document}

\maketitle
\tableofcontents

\section{课前\hfill{}\textsc{预习}}
\label{sec:org9f86bbe}
\subsection{气体}
\label{sec:orgf6bfc1c}
\subsubsection{等离子态\hfill{}\textsc{概念}}
\label{sec:org6a9dd88}
其中包含带正负电荷的离子,电子,以及少量未经电离的分子原子等,整体呈电中性
\subsubsection{气体分子动理论}
\label{sec:org368b6d2}
\begin{quote}
pV=nRT :公式:
\end{quote}
压力越低,温度越高,气体越能符合这个关系式,
任何情况下可以严格满足此关系式的气体称为理想气体

\subsubsection{气体分子运动的微观模型}
\label{sec:org0c4e01a}
\begin{itemize}
\item 气体分子本身的体积很小可以忽略
\item 气体分子不断作规则运动均匀地分布在容器中
\item 分子彼此、分子与器璧的碰撞是完全弹性的
\end{itemize}
这种假设在低压下分子间距远大于分子本身的体积时是适用的

\section{上课}
\label{sec:orgbffe346}
\subsection{课程信息}
\label{sec:org105a27d}
\subsubsection{名称}
\label{sec:org1e1fd33}
物理化学I
\subsubsection{teacher}
\label{sec:org154a8cc}
汪文栋
\subsubsection{e-mail}
\label{sec:org1820d42}
wangwd@ustc.edu.cn
\subsubsection{todo\hfill{}\textsc{TODO}}
\label{sec:orgf49b60e}
热学课本
\subsubsection{考核}
\label{sec:org65408f1}
\begin{itemize}
\item 平时成绩 40\%
\begin{itemize}
\item 测验,作业,期中
\end{itemize}
\item 期末 60\%(上下)
\end{itemize}


\subsection{绪论}
\label{sec:org3582456}
\subsubsection{化学热力学}
\label{sec:orga3c3d0b}
化学反应的方向和限度
\subsubsection{化学动力学}
\label{sec:org701b36b}
化学反应的速率和机理
\subsubsection{物质结构}
\label{sec:org65feaf8}
物质结构与性能之间的关系

\subsection{热力学基础知识}
\label{sec:org969c45c}
\subsubsection{热现象,热运动}
\label{sec:org52aa19a}
\begin{enumerate}
\item 热现象
\label{sec:org9ad9a5e}
与*宏观*物体冷热状态相关联的自然现象
\item 研究对象
\label{sec:org18b37f7}
大量粒子的集合
\end{enumerate}
\subsubsection{热力学基本定律}
\label{sec:org59aab5b}
\begin{enumerate}
\item 温度,宏观物体趋于冷热相同
\item 能量守恒
\item 自发过程,熵增加
\item 低温极限,绝对零度,熵的零点
\end{enumerate}

宏观理论广泛普适,但是不能描述微观状态

分子数密度: n
\begin{quote}
\[
n=\frac{N}{V}
\]
\end{quote}

\subsubsection{分子热运动}
\label{sec:org57e761c}
\begin{itemize}
\item 分子之间都有间隙
\item 任何宏观物体内部分子都在永不停息运动(扩散)
\item 热运动:大量分子与冷热相关的无规则运动
\end{itemize}
\subsubsection{分子间相互作用}
\label{sec:org0f028f2}
\begin{itemize}
\item 短程力
\item 吸引力
\item 排斥力
\item 排斥力增大比引力快
\end{itemize}
\begin{enumerate}
\item 公式
\label{sec:orgd2eaaf7}
略1
\item 分子的有效直径
\label{sec:org21d676b}
气体分子能够接近的最近的距离(此时有一定的排斥力)
\item 合力为零的平衡距离
\label{sec:org4580c02}
此处有一个势阱,结合能是将处于平衡距离的两个分子
分开所需要的最小能量
\item 分子间作用力的产生因素
\label{sec:org70cd749}
\begin{itemize}
\item 永久偶极作用
\item 诱导偶极作用
\item 瞬间偶极作用
\end{itemize}
\end{enumerate}
\end{document}
