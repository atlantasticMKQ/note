% Created 2019-09-23 一 09:18
% Intended LaTeX compiler: xelatex
\documentclass[11pt]{article}
\usepackage{graphicx}
\usepackage{grffile}
\usepackage{longtable}
\usepackage{wrapfig}
\usepackage{rotating}
\usepackage[normalem]{ulem}
\usepackage{amsmath}
\usepackage{textcomp}
\usepackage{amssymb}
\usepackage{capt-of}
\usepackage{hyperref}
\usepackage[scheme=plain]{ctex}
\author{MKQ}
\date{\today}
\title{Physical Chemistry 07}
\hypersetup{
 pdfauthor={MKQ},
 pdftitle={Physical Chemistry 07},
 pdfkeywords={note},
 pdfsubject={},
 pdfcreator={Emacs 24.5.1 (Org mode 9.0.3)}, 
 pdflang={English}}
\begin{document}

\maketitle
\tableofcontents

\section{卡诺循环}
\label{sec:org773f4df}
由两个绝热过程,两个可逆过程组合而成
其实做的功和吸的热都是来自于那两个等温过程
\begin{itemize}
\item 工作在高温热源和低温热源之间
\end{itemize}
\subsection{热机效率}
\label{sec:org552daa8}
取决于两个热元之间的温度差
\[
\frac{-W}{Q}=1-\frac{T_L}{T_H}
\]
如果两个温度差趋近于零,热机效率可以到达100\%
但是这种只适用于可逆的热机,实际上不是可逆过程
是理论上的最高效率
\subsection{倒开卡诺循环}
\label{sec:org96c35cf}
从低温热源吸热,把功和热传到高温热源
起到一个制冷效果
\subsubsection{冷冻效率}
\label{sec:org41cff06}
\[
\frac{T_L}{T_H-T_L}
\]
通常比热机效率高一些
\subsection{热泵}
\label{sec:org725d9b9}
循环和冷冻一样
类似于空调制热
空调制热效率高于电暖气
\section{其他的循环}
\label{sec:org45c0ccb}
\subsection{斯特林循环}
\label{sec:org0504b0a}
两个等温两个等容
效率和卡诺循环类似
\begin{itemize}
\item 是一种外燃机,在外部燃烧燃料
\end{itemize}
\subsection{Otto热机}
\label{sec:org85a51ac}
四冲程热机
绝热过程和等容过程组合而成,是内燃机
\begin{itemize}
\item 绝热压缩
\item 等容升温
\item 绝热膨胀
\item 等容降温
\end{itemize}
\section{热力学定律在实际气体}
\label{sec:org1f72d53}
\subsection{Joule-Thomson效应}
\label{sec:org55ee7bc}
利用多空塞造成一个两边的压力差
\subsubsection{节流膨胀过程}
\label{sec:org4219194}
是个等焓的过程
\subsection{气体的等焓线和转化曲线}
\label{sec:orga542a93}
\end{document}
