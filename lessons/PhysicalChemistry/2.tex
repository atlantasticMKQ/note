% Created 2019-09-04 三 11:22
% Intended LaTeX compiler: xelatex
\documentclass[11pt]{article}
\usepackage{graphicx}
\usepackage{grffile}
\usepackage{longtable}
\usepackage{wrapfig}
\usepackage{rotating}
\usepackage[normalem]{ulem}
\usepackage{amsmath}
\usepackage{textcomp}
\usepackage{amssymb}
\usepackage{capt-of}
\usepackage{hyperref}
\usepackage[scheme=plain]{ctex}
\author{MKQ}
\date{\today}
\title{Physical Chemistry 02}
\hypersetup{
 pdfauthor={MKQ},
 pdftitle={Physical Chemistry 02},
 pdfkeywords={physical-chemistry note},
 pdfsubject={},
 pdfcreator={Emacs 24.5.1 (Org mode 9.0.3)}, 
 pdflang={English}}
\begin{document}

\maketitle
\tableofcontents


\section{体系与环境}
\label{sec:orgaeb7a16}
\subsection{体系}
\label{sec:org4815413}
用人为的边界把一部分物质与其他物质分开,这部分被选定的物质叫做系统.
(System)

\begin{itemize}
\item 就是热力学的研究对象
\end{itemize}
\subsection{环境}
\label{sec:org403808f}
就是系统以外的部分,但实际上考虑和系统关系密切的部分
\subsection{系统的分类}
\label{sec:orgbc1a224}
\subsubsection{按照物质能量交换}
\label{sec:org3bca118}
\begin{itemize}
\item 敞开系统(open system)
\end{itemize}
有物质能量交换
\begin{itemize}
\item 封闭系统(closed system)
\end{itemize}
有物质交换无能量交换
\begin{itemize}
\item 孤立系统(isolated System)
\end{itemize}
没有物质能量交换

\begin{itemize}
\item 能量交换只有热和功两种形式在热力学里面
\end{itemize}
\subsubsection{按照物质的种类}
\label{sec:org5d7c96a}
\begin{itemize}
\item 单组分系统
\end{itemize}
\begin{itemize}
\item 多组分系统
\end{itemize}
\subsubsection{按照相的数目}
\label{sec:org00cc3e0}
\begin{itemize}
\item 单相系统(均相系统:例如气体)
\item 多相系统
\begin{itemize}
\item 相:物质物理化学性质相同的部分
+气体:一个相
+液体:可以有很多,按照互溶程度
+固体:一块固体为一个相
\end{itemize}
\end{itemize}
\subsection{1系统宏观性质和状态函数}
\label{sec:org4940f3f}
\subsubsection{系统性质}
\label{sec:orgc71da38}
热力学变量,宏观可以测量,分两类
\begin{itemize}
\item 有加和性:广度性质,数学上是一次函数
\item 没加和性:强度性质,多为广度性质之间的比值什么的
\item 都不是:电容电阻什么的
\end{itemize}

\subsubsection{状态函数}
\label{sec:org22cc58e}
\begin{itemize}
\item 状态:
\item 状态参数
\item 状态函数:系统经过变化,只要回到原来的状态,状态函数不变,
\end{itemize}
状态确定,状态函数就唯一确定(不同状态可能温度一样,但相同状态温度不会不一样)
\begin{itemize}
\item 状态函数的该变量与变化的途径没有关系
\item 状态函数的环路积分为零
\item 状态函数有全微分的性质
\end{itemize}
\[p=f(n,V,T)\]
\begin{itemize}
\item 二次全微分和求偏导的次序没有关系
\item MKQ可厉害啦
\item 还有个公式记在书上了
\item 老师在讲复合函数的偏微分
\end{itemize}
\[
z=f(x,y)=f[x,y(x,\alpha)]
\]
\begin{itemize}
\item 他说没有比这个更复杂的数学啦:我觉得他骗人
\end{itemize}
\begin{itemize}
\item 状态方程:状态函数间的定量关系式,说明这堆状态函数之间是联系的,
\end{itemize}
一个变化了另一个也会变
\begin{itemize}
\item 描述一个系统需要多少个变量其实热力学是不会告诉你的,需要你自己实验
\item 举了个例子;说明这玩意怎么来的
\end{itemize}
\[
f(p,V,T)=0 --> V=f(p,T)
我把它记在书上了
\]
\begin{itemize}
\item 热力学平衡态
\end{itemize}
没有宏观的离子能量流动,此时系统各个相的宏观性质都不发生变化,此时就是热力学
平hentai了\#喂
\begin{itemize}
\item 热平衡:不然会有热传导
\item 力学平衡:不然会有形变和功
\item 相平衡:不会有相的产生和消失
\item 化学平衡:化学反应不行
\end{itemize}
\begin{itemize}
\item 只有平衡态时系统宏观性质才有单值,才有状态函数
\item 稳定态(和平衡态区别):可能个部分宏观性质不变,但是可能一个相内各部分性质不同
\end{itemize}
(不光确定值,还要是确定单值)
\subsubsection{过程和途径}
\label{sec:org1a9bd95}
系统从始态到终态的整个变化
\begin{enumerate}
\item 过程
\label{sec:org946251e}
只与始态终态有关
\item 途径
\label{sec:org3b87660}
变化中间经历的一系列步骤
\begin{itemize}
\item 相同的始态终态可能经历了很多很多种不同的途径
\item 功和热是和途径有关的
\end{itemize}
\end{enumerate}
\subsubsection{常见的过程}
\label{sec:org8861db6}
等温等压绝热恒容循环
\begin{itemize}
\item 爆炸可以近似看作绝热
\end{itemize}

\begin{enumerate}
\item 可逆过程
\label{sec:org22e5ec4}
可以看做经历的每一个微小的变化都在平衡态之间进行,中间状态接近于平衡态,
也叫作准静态过程,无限缓慢,速度趋于零
\end{enumerate}
\section{热力学第零定律和温度}
\label{sec:org9bd7f1c}
\subsection{热力学第零定律}
\label{sec:orgd7fa599}
\subsubsection{温度的定义}
\label{sec:org1fc751f}
\begin{itemize}
\item 朴素的温度定义实际上是导热速度的快慢
\end{itemize}
\begin{enumerate}
\item 热平衡现象
\label{sec:org82f874b}
热力学利用热平衡状态来定义温度
不同温度的物体放在一起,过会就温度变得一样了(就不再有热传导了)
于是要求达到热平衡状态的两个物体具有一个相同状态函数,这个状态函数被定义为温度
\begin{itemize}
\item 温度相等是热平衡的充分必要条件
\item 当两个系统分别与另外一个系统达到了热平衡,那么这两个系统也达到了热平衡
\end{itemize}
\end{enumerate}
\subsubsection{温标}
\label{sec:org87f5213}
老师说这个了解一下
选取一种测温性质,这种测温性质和温度有一种线性的关系(电阻,长度,体积等)
\begin{enumerate}
\item 摄氏温标
\label{sec:org6af1b43}
必须要纯水,其中不能溶解有气体
\item 开氏温标
\label{sec:org9dd3fe3}
通过理想气体温标来实现

\item 理想气体温标
\label{sec:org7e1ae46}
\end{enumerate}

\section{理想气体}
\label{sec:orgcd99f5c}
反正老娘都会
\subsection{理想气体模型}
\label{sec:org23c9aed}
\begin{itemize}
\item 气体分子式没有尺寸的质点
\item 气体分子之间没有相互作用
\item 实际气体在温度不太低,压力不太高的时候看作理想气体没有问题
\end{itemize}
\subsection{理想气体的状态}
\label{sec:orge66d3c3}
\subsubsection{压力}
\label{sec:orga867b81}
来自于气体分子对于容器壁无休止的碰撞
\[
1 bar = 10^5 Pa
\]
\subsubsection{温度}
\label{sec:orgae26445}
果断还是选择开氏温标呀
\subsection{单一理想气体状态方程}
\label{sec:org554835e}
然后就是一堆实验,测出来一堆定律,总之归纳起来就是
\[
pV=nRT
\]
p \(\to\) 0时适用
\begin{itemize}
\item 然后就是上面那个公式的全微分
\end{itemize}
\subsection{理想气体的混合物}
\label{sec:orgc00bbda}
道尔顿分压定律,还有xxx分体积定律\ldots{}
\section{气体分子动理论}
\label{sec:org106508e}
\begin{quote}

\end{quote}
\end{document}
