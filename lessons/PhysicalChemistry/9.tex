% Created 2019-10-16 三 11:22
% Intended LaTeX compiler: xelatex
\documentclass[11pt]{article}
\usepackage{graphicx}
\usepackage{grffile}
\usepackage{longtable}
\usepackage{wrapfig}
\usepackage{rotating}
\usepackage[normalem]{ulem}
\usepackage{amsmath}
\usepackage{textcomp}
\usepackage{amssymb}
\usepackage{capt-of}
\usepackage{hyperref}
\usepackage[scheme=plain]{ctex}
\author{MKQ}
\date{\today}
\title{Physical Chemistry09}
\hypersetup{
 pdfauthor={MKQ},
 pdftitle={Physical Chemistry09},
 pdfkeywords={note},
 pdfsubject={},
 pdfcreator={Emacs 24.5.1 (Org mode 9.0.3)}, 
 pdflang={English}}
\begin{document}

\maketitle
\tableofcontents

\section{热力学第二定律}
\label{sec:org73e5340}
自发变化
\section{不可逆过程中熵变的计算}
\label{sec:org9ee5a5a}
不可逆过程在P-V图中用虚线将始态终态连接起来,但这并不代表真实的过程
然后可以把它划分为若干可逆的过程,由此来计算熵变
(绝热可逆的熵变为0)
(焓没有明确的定义)
\[
\Delta S=\int_{T_1}^{T_2} \frac{C_V}{T} dT
\]
对于等容过程
\subsection{热传导的熵变计算}
\label{sec:org9cd3613}
有宏观温度差的热传导是熵增不可逆自发过程
\subsection{气体混合熵变计算}
\label{sec:orgad4b039}
假设各种气体都膨胀到那个体积和分压,算熵变
\subsection{相变过程的计算}
\label{sec:org1073c14}
\subsubsection{可逆的相变过程}
\label{sec:orgaefb0c4}
\[
\Delta S=(\frac{Q}{T})_R =\frac{\Delta H}{T}
\]
\subsubsection{不可逆相变的计算}
\label{sec:org6d08e88}
可逆相变的负值是环境的熵变
在可逆相变的基础上设计一个循环
\subsection{化学反应过程的熵变}
\label{sec:org13e1a15}
通过设计可逆的电池反应,用标准电动势来计算
\[
\Delta_r H_m^0 =\Delta_r G_m^0 + T\Delta S_m^0
\]
\(\Delta\) G是能做的最大的有用功
\subsection{能量退降}
\label{sec:orgbd21f8c}
做功能力下降品质降低

\subsection{热源见的热传导}
\label{sec:org58b9881}
由于是热源温度不会变化,熵变大了
\section{熵的统计意义}
\label{sec:orgc7e8fae}
\[
S=k log\Omega
\]
\subsection{熵随着温度的升高必然会增大}
\label{sec:org8285629}
气体熵主要贡献者是平动,液体是转动,固体振动+电子
\[
(\frac{\partial S}{\partial T})_p =\frac{C_p}{T} >0
\]
\[
(\frac{\partial S}{\partial T})_V =\frac{C_V}{T} >0
\]
\subsection{Gibbs 自由能}
\label{sec:org2872213}
\[
dU=\delta Q+\delta W 
\]
\[
dS-\frac{\delta Q}{T} >=0
\]

于是就得到了
$\backslash$[

$\backslash$]
\subsection{Helmholtz自由能}
\label{sec:org4347327}
\[
A=U-TS
\]
等温过程中,一个封闭体系对外做的最大的功
在不断地减小
\section{Gibbs自由能}
\label{sec:org97c0344}
\begin{itemize}
\item 等温等压条件下
\end{itemize}
表示能做的最大的非体积功
\[
-\delta W_f <= -d(U+pV-TS) \mbox{或} -\delta W_f <= -d(H-TS)
\]
它做的功不大于G的减少 同时外界对它做的功不小于G的增加

\section{亥姆霍兹自由能}
\label{sec:orgffb4694}
\begin{itemize}
\item 等温条件下
\end{itemize}
描述的是不做体积功的情况
\[
A=U-TS
\]
\section{把反应设计为一个电化学反应来做非体积功}
\label{sec:org2105433}
\[
\Delta G=-nEF
\]
\subsection{各种判据}
\label{sec:org57a3c7d}
\begin{itemize}
\item 熵
\end{itemize}
\begin{center}
\begin{tabular}{ll}
熵判据 & \\
封闭系统 & 熵变大于等于热温商之和\\
封闭/绝热系统 & \\
隔离系统 & 熵永远增加\\
 & \\
\end{tabular}
\end{center}
\begin{itemize}
\item Gibbs自由能
\end{itemize}
\begin{center}
\begin{tabular}{lllll}
封闭等温等压有非体积功 & -dG=-dW\(_{\text{f}}\) & 可逆 & 如果大于是不可逆的,小于是不可能的 & \\
\hline
上面那个没体积功 & dG=0 & 可逆 & 同上 & \\
\hline
 &  &  &  & \\
\end{tabular}
\end{center}
\begin{itemize}
\item 亥姆霍兹自由能
\end{itemize}
\begin{center}
\begin{tabular}{lllll}
 &  &  &  & \\
\hline
 &  &  &  & \\
\end{tabular}
\end{center}
\subsection{\(\Delta\) G和\(\Delta\) A的计算}
\label{sec:org36c0e55}
\subsection{\(\Delta\) G的计算}
\label{sec:org1be6cb6}
\begin{itemize}
\item 定义式
\item 等温可逆非体积功
\item vant Hoff平衡箱
\item 
\end{itemize}
\[
\Delta_r G_m^0 =-RTlnK^0
\]
\section{vant Hoff平衡箱}
\label{sec:org17cceba}
\begin{itemize}
\item reaction below
\end{itemize}
\[
dD(g,p_D)+eE(g,p_E)==== fF(g,p_F)+gG(g,p_G)
\]
给定条件下未必可逆要设想可逆过程
\begin{itemize}
\item 设想一平衡箱

\item 先DE等温可逆变压变成终态压力
\item 注入DE缓慢抽出FG
\item FG变成给定压力
\end{itemize}

\[
\Delta_r G_m =-RTlnK^0 +RTlnQ
\]
\section{几个热力学函数间的关系}
\label{sec:orgd6cd78d}
\subsection{基本关系式}
\label{sec:org54276d2}
\begin{itemize}
\item H=U+PV
\item A=U-TS
\item G=H-TS
\end{itemize}
\subsection{微分关系式}
\label{sec:org2bf35c1}
\[
dU=\delta Q+\delta W
\]

\[
dS=\frac{Q_{rev}}{T}
\]
\[
dU=TdS-pdV U=U(S,V)
\]
\[
dH=dU+d(pV)=dU+pdV+Vdp
\]
\[
dH=TdS+Vdp H=H(S,p)
\]
\[
dA=dU-d(TS)=dU-TdS-SdT
\]
\[
dA=-pdV-SdT A=A(V,T)
\]
\[
dG=dH-d(TS)=Vdp-SdT G=G(p,T)
\]
\subsection{偏导数关系式}
\label{sec:org41e826e}
\[
U=U(S,V)
\]
由此推得
\[
dU=(\frac{\partial U}{\partial S})_V dS+(\frac{\partial U}{\partial V})_S dV
\]
和上面的对比一下
\[
T=(\frac{\partial U}{\partial S})_V dS
\]
\[
p=(\frac{\partial U}{\partial V})_S dV
\]
\begin{itemize}
\item 其余的就这么推
\end{itemize}
\subsection{偏导数和偏导数的关系}
\label{sec:org611bf9a}
函数z=f(x,y)在D存在且处处连续,则
\[
\frac{\partial^2 z}{\partial x \partial y}=\frac{\partial^2 z}{\partial y \partial x}
\]
由于
\[
(\frac{\partial U}{\partial S})_V =T
\]
那么
\[
(\frac{\partial T}{\partial V})_V=\frac{\partial^2 U}{\partial S \partial V}=\frac{\partial^2 U}{\partial V
 \partial S}
\]
\section{特性函数和特征变量}
\label{sec:org1b66bdc}
选择合适的独立变量,就可以从一个已知的热力学函数推导出全部的均匀系统的平衡性质
\begin{itemize}
\item 例如
\end{itemize}

\[
dG=-SdT+Vdp
\]
G是特性函数,T,p是特征变量
\end{document}
