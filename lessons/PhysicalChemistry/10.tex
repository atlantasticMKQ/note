% Created 2019-10-23 Wed 10:33
\documentclass[11pt]{article}
\usepackage[utf8]{inputenc}
\usepackage[T1]{fontenc}
\usepackage{fixltx2e}
\usepackage{graphicx}
\usepackage{grffile}
\usepackage{longtable}
\usepackage{wrapfig}
\usepackage{rotating}
\usepackage[normalem]{ulem}
\usepackage{amsmath}
\usepackage{textcomp}
\usepackage{amssymb}
\usepackage{capt-of}
\usepackage{hyperref}
\usepackage[scheme=plain]{ctex}
\author{MKQ}
\date{\today}
\title{multi-component-system}
\hypersetup{
 pdfauthor={MKQ},
 pdftitle={multi-component-system},
 pdfkeywords={note},
 pdfsubject={},
 pdfcreator={Emacs 24.5.1 (Org mode 8.3.3)}, 
 pdflang={English}}
\begin{document}

\maketitle
\tableofcontents

\section{Perface}
\label{sec:orgheadline2}
In this chapter we discuss homogeneous-system
\begin{itemize}
\item mixture
\item solution
\item dilute solution
\end{itemize}
\subsection{solution}
\label{sec:orgheadline1}
\begin{itemize}
\item solvent
\item solute
\end{itemize}
\section{partial molar quantity}
\label{sec:orgheadline4}
not all capacity natures are addtive, water and EtOH for example
\[
V_{mix}\neq V_{H_{2}O}+V_{EtOH}
\]
except the mess
\subsection{capacity nature Z}
\label{sec:orgheadline3}
\[
Z=Z(T,p,n_1,n_2,...))
\]
\[
dZ=(\frac{\partial Z}{\partial T})_{p,n_1,n_2...}dT+....
\]
\[
Z_b^' =(\frac{\partial Z}{\partial n_B})_{T,p,n_C (C\neq B)}
\]
\[
dZ=\Sum^{k}_{B=1}Z_B^' dn_B
\]
the addition of Z when add 1 mol B i the solution
\begin{itemize}
\item Z can be G H U V \ldots{}
\item while Z=G it is chemical potetial
\item partial molar quantity is strength propities
\end{itemize}

if you add components proportionally Z' won't changed
\end{document}
