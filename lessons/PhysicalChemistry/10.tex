% Created 2019-10-28 一 08:19
% Intended LaTeX compiler: xelatex
\documentclass[11pt]{article}
\usepackage{graphicx}
\usepackage{grffile}
\usepackage{longtable}
\usepackage{wrapfig}
\usepackage{rotating}
\usepackage[normalem]{ulem}
\usepackage{amsmath}
\usepackage{textcomp}
\usepackage{amssymb}
\usepackage{capt-of}
\usepackage{hyperref}
\usepackage[scheme=plain]{ctex}
\author{MKQ}
\date{\today}
\title{multi-component-system}
\hypersetup{
 pdfauthor={MKQ},
 pdftitle={multi-component-system},
 pdfkeywords={note},
 pdfsubject={},
 pdfcreator={Emacs 24.5.1 (Org mode 9.0.3)}, 
 pdflang={English}}
\begin{document}

\maketitle
\tableofcontents

\section{Perface}
\label{sec:orgbfe8429}
In this chapter we discuss homogeneous-system
\begin{itemize}
\item mixture
\item solution
\item dilute solution
\end{itemize}
\subsection{solution}
\label{sec:org4f2ceec}
\begin{itemize}
\item solvent
\item solute
\end{itemize}
\section{partial molar quantity}
\label{sec:org7cf869d}
not all capacity natures are addtive, water and EtOH for example
\[
V_{mix}\neq V_{H_{2}O}+V_{EtOH}
\]
except the mess
\subsection{capacity nature Z}
\label{sec:org5615782}
\[
Z=Z(T,p,n_1,n_2,...))
\]
\[
dZ=(\frac{\partial Z}{\partial T})_{p,n_1,n_2...}dT+....
\]
\[
Z_b^' =(\frac{\partial Z}{\partial n_B})_{T,p,n_C (C\neq B)}
\]
\[
dZ=\sum^{k}_{B=1}Z_B^' dn_B
\]
the addition of Z when add 1 mol B i the solution
\begin{itemize}
\item Z can be G H U V \ldots{}
\item while Z=G it is chemical potetial
\item partial molar quantity is strength propities
\end{itemize}

if you add components proportionally Z' won't changed
\subsection{Gibbs-Duhem 's Law}
\label{sec:orgd5c2182}
\[
\sum_{B=1}^{k}n_B dZ_B =0
\]
\section{chemical potential}
\label{sec:org699f18b}
\[
dU=(\frac{\partial U}{\partial S})_{V,n_B} dS + ...
\]
\[
\mu_B=(\frac{\partial U}{\partial n_B})_{V,p,n_C} =(\frac{\partial H}{\partial n_B})_{S,p,n_C}=(\frac{\partial G}{\partial n_B})_{T,p,n_C}=(\frac{\partial A}{\partial n_B})_{TV,n_C}
\]
\[
dG=-SdT+Vdp+\sum_{B=1}^{k}(\frac{\partial G}{\partial n_B})_{T,p,n_C (C\neq B)}d_B
\]


\section{气体混合物中各组分的化学势}
\label{sec:org8c577b2}
\subsection{单组分理想气体}
\label{sec:org93cae04}
\[
\mu=\left(\frac{\partial G}{\partial n}\right)=G_{m}
\]
\[
\mu(T,p)=\mu^{0}(T,p^{0})+RTln\left(\frac{p}^{0}}\right)
\]
\subsection{对于多组分理想气体混合物}
\label{sec:org0d1f677}
情况几乎一样,只是压力换成了分压

\subsection{单组分的实际气体的化学势}
\label{sec:orgd899f41}
\[
pV_{m}=RT\left(1+\frac{ap}{1+ap}\right)
\]
\[
\left(\frac{\partial \mu}{\partial p}\right)_{T}=V_{m}=\frac{RT}{p}\left(1+\frac{ap}{1+ap}\right)
\]
\[
\int d\mu=\int \frac{RT}{p}\left(1+\frac{ap}{1+ap}\right)dp
\]
\end{document}
