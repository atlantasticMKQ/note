% Created 2019-09-05 四 09:40
% Intended LaTeX compiler: xelatex
\documentclass[11pt]{article}
\usepackage{graphicx}
\usepackage{grffile}
\usepackage{longtable}
\usepackage{wrapfig}
\usepackage{rotating}
\usepackage[normalem]{ulem}
\usepackage{amsmath}
\usepackage{textcomp}
\usepackage{amssymb}
\usepackage{capt-of}
\usepackage{hyperref}
\usepackage[scheme=plain]{ctex}
\author{MKQ}
\date{\today}
\title{fubianhanshu 01}
\hypersetup{
 pdfauthor={MKQ},
 pdftitle={fubianhanshu 01},
 pdfkeywords={note fubianhanshu},
 pdfsubject={},
 pdfcreator={Emacs 24.5.1 (Org mode 9.0.3)}, 
 pdflang={English}}
\begin{document}

\maketitle
\tableofcontents

\section{复数的几何表示}
\label{sec:orgc3387d6}
全体复数和复平面上的点一一对应,复数z可以由平面上的一个自由向量来表示,同时还可以由向量的
幅角和长度来表示
\[
r=\sqrt{x^2 +y^2 },\tan\psi =\frac{y}{x}
\]
\begin{itemize}
\item r:复数z的模
\item \(\psi\) :是复数z的幅角
\end{itemize}
\[
r=\|z\|,\psi=Arg z
\]
\subsection{注意}
\label{sec:org67f931c}
\begin{itemize}
\item 任意复数都有无穷多的幅角,它们之间相差2n\(\pi\),于是约定,
\end{itemize}
用arg z表示一个复数在(-\(\pi\),\(\pi\)]内确定的幅角
\[
Arg z=arg z + 2n\pi
\]
\begin{itemize}
\item z=0时,幅角是没有意义的
\end{itemize}
\section{复数的三角表示}
\label{sec:org62444b0}
\[
x=r \cos\psi,y=r\sin\psi
\]
所以
\[
z=r(\cos\psi + i \sin\psi)
\]
根据欧拉公式
\[
e^{i\psi}=\cos \psi + i\sin\psi
\]
复数还可以写为指数形式
\[
z=e^{i\psi}
\]
\begin{itemize}
\item 复数相等的充要条件
\end{itemize}
\[
r_1 =r_2 ,\psi_1 =\psi_2 +2n\pi
\]
\begin{itemize}
\item 复数共轭的关系
\end{itemize}
\[
\| \bar{z} \| = \|z\| ,arg \bar{z}= - arg z,arg z \neq \pi \]                                                                 |
\end{document}
