% Created 2019-09-10 二 09:37
% Intended LaTeX compiler: xelatex
\documentclass[11pt]{article}
\usepackage{graphicx}
\usepackage{grffile}
\usepackage{longtable}
\usepackage{wrapfig}
\usepackage{rotating}
\usepackage[normalem]{ulem}
\usepackage{amsmath}
\usepackage{textcomp}
\usepackage{amssymb}
\usepackage{capt-of}
\usepackage{hyperref}
\usepackage[scheme=plain]{ctex}
\author{MKQ}
\date{\today}
\title{}
\hypersetup{
 pdfauthor={MKQ},
 pdftitle={},
 pdfkeywords={note},
 pdfsubject={},
 pdfcreator={Emacs 24.5.1 (Org mode 9.0.3)}, 
 pdflang={English}}
\begin{document}

\tableofcontents

\section{导数和XX函数}
\label{sec:org06524ea}
\begin{itemize}
\item w=f(z)在z的邻域U里有定义,z+\(\Delta\) z \(\in\) U
\end{itemize}
如果极限
\[
lim_{\Delta z -> 0} \frac{f(z+\Delta z)-f(z)}{\Delta z}
\]
存在.
就是可导.
\subsection{连续}
\label{sec:org4d00cce}
\subsection{可微}
\label{sec:org5264930}
如果f(z)在区域D中每点都可微,那f(z)在D上解析
如果在z\(_{\text{0}}\) 的某个邻域U上可微,那么f(z)在这一点是解析的(必须要有一个邻域,而不是单点)
如果在z\(_{\text{0}}\) 不是解析的,那么z\(_{\text{0}}\) 是奇点
在区域上点点可微就是区域解析
\subsection{复解析函数求导法则}
\label{sec:org5232370}
和实变量函数完全相同
\section{柯西-黎曼方程(C-R function)}
\label{sec:orgfd84492}
\subsection{定义}
\label{sec:org8bea2bc}
f(z)=u(x,y)+iv(x,y)在z=x+iy可微,等价于
\begin{itemize}
\item u,v在(x,y)二元可微
\item u,v满足C-R方程
\end{itemize}
\[
\frac{\partial u}{\partial x}=\frac{\partial v}{\partial y}
\]
\[
\frac{\partial u}{\partial y}=-\frac{\partial v}{\partial x}
\]

\subsection{性质}
\label{sec:org6d0de16}
如果满足了C-R方程,那么u,v都是调和函数
\[
\frac{\partial^2 u}{\partial^2 x}+\frac{\partial^2 v}{\partial^2 y}
\]
\[
\frac{\partial}{\partial z}=\frac{1}{2}\(\frac{\partial }{\partial x} - i\frac{\partial}{\partial y}\)
\]
\section{初等函数}
\label{sec:orge308c5b}
\subsection{指数函数}
\label{sec:orgb6a9ce2}
\[
e^z
\]
就是复的指数函数
它的微分还是它自己
\subsubsection{指数函数的性质}
\label{sec:org9eb87f5}
\begin{itemize}
\item 它的模长是e\(^{\text{x}}\) 所以说在任何地方都不会为零
\item z趋于无穷时,极限不存在
\item 相乘就是指数相加
\item e\(^{\text{z}}\) 的周期是2\(\pi\) i
\end{itemize}
\subsection{三角,双曲函数}
\label{sec:org6a64f6f}
\(\cos\) z , \(\tan\) z , \(\cot\) z \ldots{}
\subsubsection{性质}
\label{sec:org4c4ebd1}
\begin{itemize}
\item cos sin cosh sinh在C上都是解析的tan cot tanh coth在分母不为零时解析,求导性质之类的和实数下相同
\end{itemize}
\end{document}
