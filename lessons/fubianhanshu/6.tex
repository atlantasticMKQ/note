% Created 2019-09-19 四 10:27
% Intended LaTeX compiler: xelatex
\documentclass[11pt]{article}
\usepackage{graphicx}
\usepackage{grffile}
\usepackage{longtable}
\usepackage{wrapfig}
\usepackage{rotating}
\usepackage[normalem]{ulem}
\usepackage{amsmath}
\usepackage{textcomp}
\usepackage{amssymb}
\usepackage{capt-of}
\usepackage{hyperref}
\usepackage[scheme=plain]{ctex}
\author{MKQ}
\date{\today}
\title{fubianhanshu 06}
\hypersetup{
 pdfauthor={MKQ},
 pdftitle={fubianhanshu 06},
 pdfkeywords={note},
 pdfsubject={},
 pdfcreator={Emacs 24.5.1 (Org mode 9.0.3)}, 
 pdflang={English}}
\begin{document}

\maketitle
\tableofcontents

\section{柯西积分公式}
\label{sec:org56422fa}
\subsection{定理}
\label{sec:orgdd5b210}
设函数f(z)在闭路C及其围成的区域是解析的,
那么对于任意的z in D 都有
\[
f(z)=\frac{1}{2\pi i}\int_C \frac{f(\psi)}{\psi -z}dz
\]

\subsection{证明}
\label{sec:org9760876}
\(\forall\) z in D 有z的邻域|\(\psi\) -z|<\tho
要求邻域全部在D上,那么
\[
\int_C \frac{f(\psi)}{\psi -z}d\psi = \int_T \frac{f(\psi)}{\psi -z}d\psi
\]
T是那个充分小邻域的边界
已知
\[
\int_C \frac{f(z)}{\psi -z}d\psi =2\pi i f(z)
\]
再根据长大不等式,这个和上面那个差一个无穷小量
\section{解析函数的导数}
\label{sec:orgd14d672}
\subsection{定理}
\label{sec:org6e6a3bc}
条件同上一个定理
那么f(z)任意阶可导,还可微
\[
f^{(n)}(z)=\frac{n!}{2\pi i}\int_C \frac{f(\psi)}{(\psi -z)^{n+1}}d\psi
\]
\end{document}
