% Created 2019-09-28 六 16:33
% Intended LaTeX compiler: xelatex
\documentclass[11pt]{article}
\usepackage{graphicx}
\usepackage{grffile}
\usepackage{longtable}
\usepackage{wrapfig}
\usepackage{rotating}
\usepackage[normalem]{ulem}
\usepackage{amsmath}
\usepackage{textcomp}
\usepackage{amssymb}
\usepackage{capt-of}
\usepackage{hyperref}
\usepackage[scheme=plain]{ctex}
\author{MKQ}
\date{\today}
\title{}
\hypersetup{
 pdfauthor={MKQ},
 pdftitle={},
 pdfkeywords={note},
 pdfsubject={},
 pdfcreator={Emacs 24.5.1 (Org mode 9.0.3)}, 
 pdflang={English}}
\begin{document}

\tableofcontents

\section{复变函数的复习}
\label{sec:orgd1a5c89}
\subsection{共轭运算的公式}
\label{sec:org6f7b68d}
\[
\overline{z_1 \pm z_2}=\overline{z_1}+\overline{z_2}
\]
\[
\overline{z_1 \cdot z_2}=\overline{z_1} \cdot \overline{z_2}
\]
\[
\overline{(\frac{z_1}{z_2})}=\frac{\overline{z_1}}{\overline{z_2}}
\]
\[
z\overline{z}=Re(z)^2 +Im(z)^2 =|z|^2
\]
\subsection{其他}
\label{sec:org00d76e7}
\subsubsection{一个实数的共轭是她本身(用于证明)}
\label{sec:orgb9f1d20}
\subsubsection{实系数多项式的根共轭存在}
\label{sec:orga1301bc}
假设z\(_{\text{0}}\)是n次多项式
\[
P(z)=z^n +a_1 z^{n-1}+a_2 z^{n-2} +...+a_{n-1}z +a_n
\]
的根,其中各个系数a\(_{\text{1}}\) ,a\(_{\text{2}}\) ,a\(_{\text{3}}\) \ldots{}a\(_{\text{n}}\) 都是实数,由共轭复数的性质有:
\begin{equation}
\begin{aligned}
P(\overline{z_0}) &=(\overline{z_0})^n +a_1 (\overline{z_1})^{n-1} + ... +a_{n-1}\overline{z_0} + a_n \\
&=(\overline{z_0})^n +\overline{a_1} (\overline{z_1})^{n-1} + ... + \overline{a_{n-1}} \overline{z_0} +\overline{a_n} \\
&=\overline{(z_0)^n + a_1 (z_0)^{n-1} + a_2 (z_0)^{n-2}+...+a_{n-1}z_0 + a_n}
&=\overline{P(z_0)}
&=0
\end{aligend}
\end{equation}
\subsection{复数模长的不等式}
\label{sec:orge482f80}
\begin{cases}
\,|Re(z)|	\\
\,|Im(z)|
\end{cases}
\(\le\) |z| \(\le\) |Re(z)|+|Im(z)|

$\backslash$|z\(_{\text{1}}\)|-|z\(_{\text{2}}\) ||\(\le\) |z\(_{\text{1}}\) \textpm{} z\(_{\text{2}}\) | \(\le\) | z\(_{\text{1}}\)| +|z\(_{\text{2}}\)|

$\backslash$| z\(_{\text{1}}\) + z\(_{\text{2}}\) +\ldots{}+z\(_{\text{n}}\)| \(\le\) |z\(_{\text{1}}\)| + |z\(_{\text{2}}\)| +\ldots{}+|z\(_{\text{n}}\)|
\end{document}
