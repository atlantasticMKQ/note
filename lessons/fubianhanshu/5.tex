% Created 2019-09-19 四 09:47
% Intended LaTeX compiler: xelatex
\documentclass[11pt]{article}
\usepackage{graphicx}
\usepackage{grffile}
\usepackage{longtable}
\usepackage{wrapfig}
\usepackage{rotating}
\usepackage[normalem]{ulem}
\usepackage{amsmath}
\usepackage{textcomp}
\usepackage{amssymb}
\usepackage{capt-of}
\usepackage{hyperref}
\usepackage[scheme=plain]{ctex}
\author{MKQ}
\date{\today}
\title{fubianhanshu 05}
\hypersetup{
 pdfauthor={MKQ},
 pdftitle={fubianhanshu 05},
 pdfkeywords={note},
 pdfsubject={},
 pdfcreator={Emacs 24.5.1 (Org mode 9.0.3)}, 
 pdflang={English}}
\begin{document}

\maketitle
\tableofcontents

\section{复变函数的积分}
\label{sec:orga6a37b8}
\subsection{定义}
\label{sec:org5734285}
\[
\int_c f(z)dz=\int_c (u+iv)(dx+idy)
\]
更类似于曲线积分(II)
\subsection{定理}
\label{sec:orgdce8e61}
\[
f(z)=u(x,y)+iv(x,y)
\]
在C上连续,则积分
\[
\int_C f(z)dz
\]
存在
\subsection{解法}
\label{sec:org7c2f9dc}
参数化,求解
拆成好几段,把每一段都参数化
\[
I=\int_C \frac{dz}{(z-a)^n}
\]
C:a为中心,R为半径的圆
\begin{cases}
I=0 (n\neq 1) \\
I=2\pi i(n=1)
\end{cases}
\subsection{长大不等式}
\label{sec:org8e745d2}
\[
[\int_C f(z)dz] <= \int_C [f(z)][dz] <= Ml
\]
\begin{itemize}
\item M:f(z)在C上最大值
\item l:C弧长
\end{itemize}
\subsection{柯西积分定理}
\label{sec:org0112f8d}
D由闭合回路C围成的单连通区域f(z)在\bar{D}=D+C
上解析,那么,
\[
\int_C f(z)dz=0
\]
\subsubsection{推论}
\label{sec:org3218e6e}
设f(z)在单联通区域D上解析,此时D内任一曲线
\[
\int_C f(z)dz=0
\]
\subsubsection{推论}
\label{sec:org2f15e5c}
设f(z)在单联通区域D解析,C是任一简单曲线在D内,
那么积分结果不依赖于C,仅仅取决于起点终点
\end{document}
