% Created 2019-09-17 二 08:41
% Intended LaTeX compiler: xelatex
\documentclass[11pt]{article}
\usepackage{graphicx}
\usepackage{grffile}
\usepackage{longtable}
\usepackage{wrapfig}
\usepackage{rotating}
\usepackage[normalem]{ulem}
\usepackage{amsmath}
\usepackage{textcomp}
\usepackage{amssymb}
\usepackage{capt-of}
\usepackage{hyperref}
\usepackage[scheme=plain]{ctex}
\author{MKQ}
\date{\today}
\title{fubianhanshu 05}
\hypersetup{
 pdfauthor={MKQ},
 pdftitle={fubianhanshu 05},
 pdfkeywords={note},
 pdfsubject={},
 pdfcreator={Emacs 24.5.1 (Org mode 9.0.3)}, 
 pdflang={English}}
\begin{document}

\maketitle
\tableofcontents

\section{复变函数的积分}
\label{sec:orga569356}
\subsection{定义}
\label{sec:org1eebab2}
\[
\int_c f(z)dz=\int_c (u+iv)(dx+idy)
\]
更类似于曲线积分(II)
\subsection{定理}
\label{sec:org3541f69}
\[
f(z)=u(x,y)+iv(x,y)
\]
在C上连续,则积分
\[
\int_C f(z)dz
\]
存在
\subsection{解法}
\label{sec:orgefb06e9}
参数化,求解
拆成好几段,把每一段都参数化
\[
I=\int_C \frac{dz}{(z-a)^n}
\]
C:a为中心,R为半径的圆
\begin{cases}
I=0 (n\(\neq\) 1) \\
I=2\(\pi\) i(n=1)
\end\{cases
\}
\subsection{长大不等式}
\label{sec:org51570bd}
\[
\int_C f(z)dz <= \int_C [f(z)][dz] <= Ml
\]
\begin{itemize}
\item M:f(z)在C上最大值
\item l:C弧长
\end{itemize}
\end{document}
