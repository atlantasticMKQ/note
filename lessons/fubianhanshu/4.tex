% Created 2019-09-12 四 10:16
% Intended LaTeX compiler: xelatex
\documentclass[11pt]{article}
\usepackage{graphicx}
\usepackage{grffile}
\usepackage{longtable}
\usepackage{wrapfig}
\usepackage{rotating}
\usepackage[normalem]{ulem}
\usepackage{amsmath}
\usepackage{textcomp}
\usepackage{amssymb}
\usepackage{capt-of}
\usepackage{hyperref}
\usepackage[scheme=plain]{ctex}
\author{MKQ}
\date{\today}
\title{FUBIAN 04}
\hypersetup{
 pdfauthor={MKQ},
 pdftitle={FUBIAN 04},
 pdfkeywords={note},
 pdfsubject={},
 pdfcreator={Emacs 24.5.1 (Org mode 9.0.3)}, 
 pdflang={English}}
\begin{document}

\maketitle
\tableofcontents

\section{上节课}
\label{sec:org1e024bb}
将三角函数,e\(^{\text{x之类的推广到了复平面,还有sinh,cosh}}\)
\subsection{基本性质}
\label{sec:org6645d00}
\subsubsection{sin z cosz sinhz cosh的周期}
\label{sec:orgb3a7d72}
\begin{itemize}
\item sinz cosz:2\(\pi\)
\item sinhz coshz:2\(\pi\) i
\end{itemize}
\subsubsection{这些函数在复平面的零点}
\label{sec:orged11124}
\begin{itemize}
\item sinz:n\(\pi\) (实数轴上的)
\end{itemize}
\[
e^{iz}=e^{-iz} \rightarrow e^{iz}=+/- 1
\]
\[
e^{-y}=1(模长) e^{ix}=+/-1
\]
\begin{itemize}
\item cosz:(n+1/2\(\pi\))
\item sinhz:n\(\pi\) i
\item coshz:(n+1/2\(\pi\))i
\end{itemize}
\subsubsection{所有三角双曲恒等式在复平面上都成立}
\label{sec:org15560c2}
\subsubsection{这些函数在复平面上是无界的}
\label{sec:orgfb3ef5c}
\begin{enumerate}
\item 例题
\label{sec:orgc1f8971}
cosz的实部虚部还有模长
\[
z=x+iy,cosz=\frac{e^{i(x+iy)}+e^{-i(x+iy)}}{2}
\]
然后依次展开
\[
cosx\frac{e^y +e^{-y}}{2}+sinx\frac{e^y +e^{-y}}{2}i
\]
\end{enumerate}
\subsection{对数函数}
\label{sec:orgf260501}
\[
e^w =z \neq 0 \rightarrow w=Ln z
\]
\[
e^w =z=|z|e^{iArgz}
\]
\[
Ln z=ln|z|+iArgz
\]
\[
ln z=ln|z|+iargz
\]
\subsubsection{栗子}
\label{sec:org8ae44fa}
\end{document}
