% Created 2019-09-24 二 09:56
% Intended LaTeX compiler: xelatex
\documentclass[11pt]{article}
\usepackage{graphicx}
\usepackage{grffile}
\usepackage{longtable}
\usepackage{wrapfig}
\usepackage{rotating}
\usepackage[normalem]{ulem}
\usepackage{amsmath}
\usepackage{textcomp}
\usepackage{amssymb}
\usepackage{capt-of}
\usepackage{hyperref}
\usepackage[scheme=plain]{ctex}
\author{MKQ}
\date{\today}
\title{fubianhanshu 07}
\hypersetup{
 pdfauthor={MKQ},
 pdftitle={fubianhanshu 07},
 pdfkeywords={note},
 pdfsubject={},
 pdfcreator={Emacs 24.5.1 (Org mode 9.0.3)}, 
 pdflang={English}}
\begin{document}

\maketitle
\tableofcontents

\section{解析函数与调和函数}
\label{sec:org5a13847}
\begin{itemize}
\item 定义:实二次函数u(x,y)在D内二阶连续可导(C\(^{\text{2}}\) )而且在D内满足拉普拉斯方程
\end{itemize}
\[
\frac{\partial^2 u }{\partial x^2}+\frac{\partial^2 u}{\partial y^2}=0
\]
\subsection{定理}
\label{sec:org3331ca0}
设f(z)=u(x,y)+iv(x,y)在D内解析,那么它的实部虚部都是D内调和函数
\subsection{定理}
\label{sec:org948d9c6}
假设f(z)是解析函数,且f'(z)\(\neq\) 0
那么等值曲线族
\[
u(x,y)=K_1
\]


和
\[
v(x,y)=K_2
\]
在公共点上永远正交
\subsection{定理}
\label{sec:orgfb8df35}
给一个实部,能找到对应的虚部

设u(x,y)是单联通区域D内的调和函数,那么
\[
v(x,y)=\int_{(x_0,y_0)}^{x,y} -\frac{\partial u}{\partial y}dx+ \frac{\partial u}{\partial x}dy+C
\]
使得f(z)在D解析,(x,y)是任意一点,(x\(_{\text{0}}\) ,y\(_{\text{0}}\) )是一个定点
\end{document}
