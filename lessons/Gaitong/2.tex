% Created 2019-09-11 三 16:36
% Intended LaTeX compiler: xelatex
\documentclass[11pt]{article}
\usepackage{graphicx}
\usepackage{grffile}
\usepackage{longtable}
\usepackage{wrapfig}
\usepackage{rotating}
\usepackage[normalem]{ulem}
\usepackage{amsmath}
\usepackage{textcomp}
\usepackage{amssymb}
\usepackage{capt-of}
\usepackage{hyperref}
\usepackage[scheme=plain]{ctex}
\author{MKQ}
\date{\today}
\title{Gaitong 02}
\hypersetup{
 pdfauthor={MKQ},
 pdftitle={Gaitong 02},
 pdfkeywords={note},
 pdfsubject={},
 pdfcreator={Emacs 24.5.1 (Org mode 9.0.3)}, 
 pdflang={English}}
\begin{document}

\maketitle
\tableofcontents

\section{条件概率}
\label{sec:org6d7f063}
已知事件B发生,在这个条件下讨论A发生的概率,样本空间被限制在一定范围内了
\subsection{注}
\label{sec:org4fb7e56}
下次概率统计课请务必早来然后坐得靠前一些
\subsection{乘法定理}
\label{sec:orgee187d3}
\[
P(AB)=P(A|B)P(B)=P(B|A)P(A)
\]
\[
P(A_1 A_2 A_3 ... A_n )=P(A_1 )P(A_2 |A_1 )P(A_3 | A_2 )...P(A_{n-1}|A_n )
\]
\subsection{事件的独立性}
\label{sec:org29c5567}
如果事件的交发生的概率等于两事件发生概率的乘积,那么这两个事件相互独立
\[
P(A|B)=P(A)
\]
那么事件AB相互独立
此外还有多个事件的独立,类似于乘法原理的定义
\subsubsection{注意}
\label{sec:org582d6c3}
如果事件两两独立,那么难说是否他们相互独立
\end{document}
