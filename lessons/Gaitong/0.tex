% Created 2019-09-04 三 15:19
% Intended LaTeX compiler: xelatex
\documentclass[11pt]{article}
\usepackage{graphicx}
\usepackage{grffile}
\usepackage{longtable}
\usepackage{wrapfig}
\usepackage{rotating}
\usepackage[normalem]{ulem}
\usepackage{amsmath}
\usepackage{textcomp}
\usepackage{amssymb}
\usepackage{capt-of}
\usepackage{hyperref}
\usepackage[scheme=plain]{ctex}
\author{MKQ}
\date{\today}
\title{概率论与数理统计}
\hypersetup{
 pdfauthor={MKQ},
 pdftitle={概率论与数理统计},
 pdfkeywords={note},
 pdfsubject={},
 pdfcreator={Emacs 24.5.1 (Org mode 9.0.3)}, 
 pdflang={English}}
\begin{document}

\maketitle
\tableofcontents

\section{课程信息}
\label{sec:org7f97901}
\subsection{教材}
\label{sec:org0b3d973}
据说要整一个自编的习题册
\begin{itemize}
\item <概率论与数理统计习题集>:去找教材科买 2r/本 :TODO:
\end{itemize}

还要整一个参考书
\begin{itemize}
\item <概率论>:科学出版社
\item <数理统计>:科学出版社
\item 学校主页上还有管理学院:精品课程,然后里面还有电子版的教案(好绝望)
\item stat.ustc.edu.cn
\item fisher.ustc.edu.cn(里面还有历年的试卷)
\end{itemize}
\subsection{teacher}
\label{sec:org6dbd095}
\begin{itemize}
\item 刘杰
\item jiel@ustc.edu.cn
\item 管科楼1015
\item 63607249
\end{itemize}
\subsection{QQ群}
\label{sec:org7a78288}
435975129
\subsection{考核}
\label{sec:org9277713}
\begin{itemize}
\item 没有期中考只有期末考,占比大概70\%
\item 平时作业30\%
\item 课上还有随机的测试,现场回答出来要+2分(不扣分)
\item 10\%的波动范围
\end{itemize}
\subsection{其他}
\label{sec:org6496ab6}
是学校48门基础课之一,可厉害了,是国家精品课程,精品资源共享课
爱课程上面有资料
还有国家级教育团队
数据挖掘,大数据都很用的
\section{统计学(是一级学科哦}
\label{sec:org5e3750b}
是数学还是艺术(感觉这个大猪蹄子又在骗我)
\subsection{赌博掷硬币}
\label{sec:org78cd1f8}
费马:掷到一半被打断-赌资该怎么分配
\subsection{拉普拉斯?}
\label{sec:org9a31c44}
建立了概率统计的公理化体系
\subsection{数据分析,大数据}
\label{sec:org5be7d7b}
\subsection{要举个例子}
\label{sec:orgc6ef89e}
关于出生率男女比例?
又在说某个法国人做了个统计(你科5:1)法国22:21;
巴黎25:24
\subsubsection{人口普查统计数据}
\label{sec:org55e5831}
\begin{itemize}
\item 1953 106:100
\item 1964 104:100
\item 1982 106:100
\item 1990 106.6:100
\item 2000 106.74:100
\item 2010 105.2:100
\item 所以这不是例题\ldots{}那我敲上来这个干吗\ldots{}
\end{itemize}
\subsection{然后又要举个例子}
\label{sec:org7217ed7}
这次我才不抄了呢\ldots{}
反正是抽烟伤身体
饮酒人数(目测是要搞相关度)
烟雾报警器\ldots{}(得和老爸安利一下
讲了一堆和寿命相关的东西
(啊这个老师有毒\ldots{}
<统计与真理>
\subsection{还有一个例子}
\label{sec:orgd11d988}
\begin{itemize}
\item 其实我觉得这个老师还是很不错的<在一本正经地幽默>
\end{itemize}
华裔妇女中秋节前后一周的死亡率<心理效应的存在>
\begin{itemize}
\item 多个子女里面老大最聪明<果然老爸是傻子>
\item 成就最高的是最小的那个
\item 如果生了女孩允许二胎那么男女比例会不会彻底失调?不会<但这个问题好讽刺
\end{itemize}
\subsection{投针测\(\pi\)}
\label{sec:org29afaca}
啊,他是投一堆针到平行线之间,然后因为针会旋转,然后可以测出\(\pi\)
\subsection{又一个例子}
\label{sec:orgc45a607}
三个门,分别是车,羊,羊,选一个,然后主持人说其中一扇不是车,那么我需要换一个门吗(感觉我不用呀)
\begin{itemize}
\item 但是选中时是1/3的概率,确定了一个不是之后,剩下一个变成了1/2,然后要换?
\end{itemize}
\section{概率}
\label{sec:orgb2091c6}
就是事件发生的可能性,介于0-1之间的一个数字
\subsection{试验,事件}
\label{sec:orgc85ce76}
\begin{itemize}
\item 试验:就是一种随机现象,某种情况可能发生或者不发生,或者多种情况发生
\item 样本空间: \(\Omega\) 试验中一切可能的结果组成的集合
\end{itemize}
\end{document}
