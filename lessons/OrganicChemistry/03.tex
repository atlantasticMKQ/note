% Created 2019-09-17 二 11:13
% Intended LaTeX compiler: xelatex
\documentclass[11pt]{article}
\usepackage{graphicx}
\usepackage{grffile}
\usepackage{longtable}
\usepackage{wrapfig}
\usepackage{rotating}
\usepackage[normalem]{ulem}
\usepackage{amsmath}
\usepackage{textcomp}
\usepackage{amssymb}
\usepackage{capt-of}
\usepackage{hyperref}
\usepackage[scheme=plain]{ctex}
\author{MKQ}
\date{\today}
\title{Organic Chemistry 03}
\hypersetup{
 pdfauthor={MKQ},
 pdftitle={Organic Chemistry 03},
 pdfkeywords={note},
 pdfsubject={},
 pdfcreator={Emacs 24.5.1 (Org mode 9.0.3)}, 
 pdflang={English}}
\begin{document}

\maketitle
\tableofcontents

\section{脂环烃的分类,异构,命名}
\label{sec:org5dd73b1}
\subsection{最小的环炔}
\label{sec:org8bc02f1}
环壬炔
\subsection{异构}
\label{sec:orga73c198}
\subsection{脂环烃的命名}
\label{sec:org47c017c}
\subsubsection{单环烃}
\label{sec:org72b3bdb}
让取代基的位次最小,优先满足双键,三键
\subsubsection{环比较简单}
\label{sec:orgf39c457}
可以用环作取代基
\subsubsection{顺反取代}
\label{sec:org850f3a4}
\subsubsection{双环烷烃}
\label{sec:orgf01eb91}
\begin{itemize}
\item 联环,可以把一个环作为取代基
\item 桥环(包括稠环)
\end{itemize}
取代基+母体,例如:二环[4.3.0]壬烷(一个五并六)
编号从一个桥头开始,走最长的桥,到另一个桥头,再编回来,优先双键三键
\begin{itemize}
\item 螺环,编号从螺原子旁边的碳开始,先较小的环,括号内也是先小环
\end{itemize}
4-螺[2.6]壬烯
\subsubsection{多环烃}
\label{sec:orgb50eeca}
\section{化学反应}
\label{sec:orgf02cb28}
三元环对于氧化剂稳定,但是对亲电加成试剂不稳定
和烯烃有相似也有不同
\subsection{对于中环7-10元环}
\label{sec:orgc75acc1}
由于半径不足够大,中间的氢会有排斥
合成比较困难
\subsection{分子内的作用力}
\label{sec:org7467f2e}
\begin{itemize}
\item 角张力,与正常键角的偏差
\item 扭转张力,与最稳定构象的偏差
\item 空间张力,小于范德华半径的排斥力
\item 偶极作用力
\end{itemize}
\end{document}
