% Created 2019-09-19 四 14:52
% Intended LaTeX compiler: xelatex
\documentclass[11pt]{article}
\usepackage{graphicx}
\usepackage{grffile}
\usepackage{longtable}
\usepackage{wrapfig}
\usepackage{rotating}
\usepackage[normalem]{ulem}
\usepackage{amsmath}
\usepackage{textcomp}
\usepackage{amssymb}
\usepackage{capt-of}
\usepackage{hyperref}
\usepackage[scheme=plain]{ctex}
\author{MKQ}
\date{\today}
\title{Data Structure 05}
\hypersetup{
 pdfauthor={MKQ},
 pdftitle={Data Structure 05},
 pdfkeywords={note},
 pdfsubject={},
 pdfcreator={Emacs 24.5.1 (Org mode 9.0.3)}, 
 pdflang={English}}
\begin{document}

\maketitle
\tableofcontents

\section{链表}
\label{sec:org8ef5ef8}
\subsection{单链表}
\label{sec:org71fcd45}
\subsection{循环链表}
\label{sec:org6cec08a}
\subsection{双向链表}
\label{sec:org5513580}
\subsection{静态链表}
\label{sec:org2cb3418}
不在教学大纲内
用数组来模拟链表,有的语言不提供指针
定义一个很大的数组,作为存储池,
一共有两条链,一条保存空闲节点,一条保存数据
其中备用节点链如果要插入的话,用头插的办法
\section{有序表}
\label{sec:org84b55ca}
约定升序排列为正
\subsection{在有序表里面插入一个元素}
\label{sec:orgcc4bab1}
\subsection{两个有序表的归并}
\label{sec:org0b1d345}
\end{document}
