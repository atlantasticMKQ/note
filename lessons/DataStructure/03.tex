% Created 2019-09-12 四 14:47
% Intended LaTeX compiler: xelatex
\documentclass[11pt]{article}
\usepackage{graphicx}
\usepackage{grffile}
\usepackage{longtable}
\usepackage{wrapfig}
\usepackage{rotating}
\usepackage[normalem]{ulem}
\usepackage{amsmath}
\usepackage{textcomp}
\usepackage{amssymb}
\usepackage{capt-of}
\usepackage{hyperref}
\usepackage[scheme=plain]{ctex}
\author{MKQ}
\date{\today}
\title{DataStructure 03}
\hypersetup{
 pdfauthor={MKQ},
 pdftitle={DataStructure 03},
 pdfkeywords={note},
 pdfsubject={},
 pdfcreator={Emacs 24.5.1 (Org mode 9.0.3)}, 
 pdflang={English}}
\begin{document}

\maketitle
\tableofcontents

\section{课前}
\label{sec:org7682b89}
\begin{itemize}
\item 定义
\item 表示
\item 实现
\item 应用
\end{itemize}
\subsection{线性表的抽象数据类型定义}
\label{sec:org23266e8}
\section{顺序表示和实现}
\label{sec:org64dacc0}
\subsection{顺序表}
\label{sec:org3cf7e3d}
顺序结构存储的线性表
\begin{verbatim}
#define LIST_INIT_SIZE 100
#define LIST_INC_SIZE 10
typedef struct
{
    ElemType    *elem;     //首地址
    int         length;    //数组长度
    int         listsize;  //总长
    int         incsize;   //增量
}
//初始化
status initList(SqList &L);
//返回首次出现的位序,时间复杂度O(N)
int locateElem(SqList L,ElemType e);
//获取某个下标的函数
status getElem(SqList L,int i,ElemType &e);
//对于顺序表,获取一个元素非常容易

//销毁顺序表
status DestroyList(SqList &q);

//插入元素
status listInsert(SqList &L,int i,ElemType e);
//删除元素,同时返回该元素的值
status elemDelete(SqList &L,int i,ElemType &e);
//对每个元素调用visit函数
status listTravserse(SqList L,void (* visit)(ElemType e));
\end{verbatim}
\end{document}
