% Created 2019-09-03 二 15:34
% Intended LaTeX compiler: xelatex
\documentclass[11pt]{article}
\usepackage{graphicx}
\usepackage{grffile}
\usepackage{longtable}
\usepackage{wrapfig}
\usepackage{rotating}
\usepackage[normalem]{ulem}
\usepackage{amsmath}
\usepackage{textcomp}
\usepackage{amssymb}
\usepackage{capt-of}
\usepackage{hyperref}
\usepackage{ctex}
\date{\today}
\title{}
\hypersetup{
 pdfauthor={},
 pdftitle={},
 pdfkeywords={},
 pdfsubject={},
 pdfcreator={Emacs 24.5.1 (Org mode 9.0.3)}, 
 pdflang={English}}
\begin{document}

\tableofcontents

\section{成绩评定}
\label{sec:org90e96e1}
\subsection{期末65-70}
\label{sec:org60d538c}
\subsection{实验+报告 20-25}
\label{sec:org5b3a171}
\subsection{作业 10 (每周二交作业<不能迟>)}
\label{sec:org8337d4a}
\section{上课啦}
\label{sec:org19f163f}
\subsection{参考书 清华 严蔚敏}
\label{sec:org3460792}
\subsection{教材是科大的那本}
\label{sec:org23cecda}
\section{第一章}
\label{sec:orgd194043}
据说比较抽象
\subsection{范畴}
\label{sec:org79d0090}
数据结构讨论问题的范畴
\begin{itemize}
\item 不能用公式方程等求解的非数值问题
\item 解决方法
\begin{itemize}
\item 分析设计的数据
\item 确定数据之间的联系
\item 数据在计算机中的组织方式(组织结构)
\item 设计解决的算法
\item 算法的时间性能,空间性能
\end{itemize}
\end{itemize}

\subsection{举例}
\label{sec:orgacce55c}
\begin{itemize}
\item 图书检索系统---线性
\item 对弈---树状
\item 地图染色---网状
\end{itemize}
\subsection{概念和术语}
\label{sec:orgad55222}
\subsubsection{Data}
\label{sec:orgff9b047}
对于客观事物的符号表示,可以输到计算机里面的那种
声音图像\ldots{}
\subsubsection{DataElement}
\label{sec:org0e0b493}
数据的基本单位,还可以称为节点,项目之类的
\subsubsection{DataItem}
\label{sec:orge8dfb6d}
数据元素的分量---数据项
\subsubsection{Set(集合)}
\label{sec:org5d4760d}
同类型值的聚合 
\begin{itemize}
\item 值不能完全相等
\item 次序任意
\item 表示方法
\begin{itemize}
\item 罗列
\item 规则
\end{itemize}
\end{itemize}
\subsubsection{DataObject}
\label{sec:org8ad974a}
和问题有关的那种,性质相同的数据元素集合
\subsubsection{DataStructure}
\label{sec:orgc2105a5}
未定义,差不多理解就好了
\begin{itemize}
\item 数据的逻辑结构
\begin{itemize}
\item DS=(D,S)
\begin{itemize}
\item D:一个数据对象
\item S:数据之间存在的多种联系,是数据元素之间关系元组的集合
\end{itemize}
\end{itemize}
\end{itemize}
\begin{quote}
所谓序偶:
X->Y 这个关系记为 <X,Y> X是Y的直接前驱,Y是X的直接后继
X--Y 这种对称的关系记作(X,Y)
\end{quote}
\begin{itemize}
\item 数据的储存结构
\item 数据的运算集合
\end{itemize}

\begin{enumerate}
\item 存储技术
\label{sec:org594962f}
\begin{itemize}
\item 顺序存储:数组
\item 链式存储:链表
\item 散列存储
\item 索引存储
\end{itemize}
\item 运算集合
\label{sec:orgea9e38d}
\begin{itemize}
\item 接口的定义
\item 运算集合的实现
\end{itemize}
\end{enumerate}
\subsubsection{DataType}
\label{sec:org459e943}
\begin{itemize}
\item 数据值的特征
\item 数据机内编码方式
\item 数据占用空间大小
\item 对于数据的运算集
\end{itemize}
\subsubsection{AbstractDataType}
\label{sec:org5471520}
一个数据类型还有定义在上面的一组操作集

(D,S,P)

\begin{itemize}
\item D 数据对象
\item S 数据结构
\item P 对于数据类型的操作
\end{itemize}

外部使用与内部实现相分离
\begin{verbatim}
ADT抽象类型名{
数据对象:<定义>
数据关系:<定义>
基本操作:<定义>
}ADT类型名
\end{verbatim}
\end{document}
