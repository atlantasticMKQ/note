% Created 2019-10-08 二 14:25
% Intended LaTeX compiler: xelatex
\documentclass[11pt]{article}
\usepackage{graphicx}
\usepackage{grffile}
\usepackage{longtable}
\usepackage{wrapfig}
\usepackage{rotating}
\usepackage[normalem]{ulem}
\usepackage{amsmath}
\usepackage{textcomp}
\usepackage{amssymb}
\usepackage{capt-of}
\usepackage{hyperref}
\usepackage[scheme=plain]{ctex}
\author{MKQ}
\date{\today}
\title{天知道今天讲的是啥我标题该是啥}
\hypersetup{
 pdfauthor={MKQ},
 pdftitle={天知道今天讲的是啥我标题该是啥},
 pdfkeywords={note},
 pdfsubject={},
 pdfcreator={Emacs 24.5.1 (Org mode 9.0.3)}, 
 pdflang={English}}
\begin{document}

\maketitle
\tableofcontents

\section{栈和队列}
\label{sec:org9acebd3}
\subsection{队列(Queue)}
\label{sec:org0b4edd7}
\subsubsection{队列的定义}
\label{sec:orgd26df6a}
在表尾插入在表尾删除,是一种特殊的线性表
\begin{itemize}
\item 特点:先进先出(FIFO)
\end{itemize}
\subsubsection{抽象数据类型定义}
\label{sec:org0f394e4}
\begin{verbatim}
ADT QUEUE{
D={a|a\in ElemSet,i=1,2,...n,n>=o}
R{<a_{i-1},a_i>|a_i \in D,i=2,...,n}

InitQueue(&Q)
DestroyQueue(&Q)
ClearQueue(&Q)
QueueEmpty(Q)
QueueLength(Q)
GetHead(Q,&a)
EnQueue(Q,a)
DeQueue(Q,&a)
}ADT Queue
\end{verbatim}
\subsubsection{链队列}
\label{sec:org20d6844}
整两个指针,一个指向头,一个指向尾
\subsubsection{循环队列}
\label{sec:org6f06aad}
循环链表,只留一个指针指向头部
\begin{enumerate}
\item 顺序结构实现
\label{sec:orgecbf1d1}
日哦,这个逊爆了
大致就是留一个指针指着下一个空位,一个指着头,然后假上溢时,
\begin{itemize}
\item 整体左移
\item 把尾指针指向数组头
\end{itemize}
\item 队满条件
\label{sec:orgd34a421}
\begin{itemize}
\item 如果尾指针和头指针差一个,那就满了,这样牺牲了一个储存节点
\item 如果相等的话,那么是队列空的意思
\end{itemize}
\end{enumerate}
\end{document}
